\begin{problem}[25pts]
    Give examples of a formal power series
\[
\sum_{n=0}^{\infty} c_n x^n
\]
centered at \(0\) with radius of convergence \(1\), which
\begin{enumerate}
\item[(a)] diverges at both \(x=1\) and \(x=-1\);
\item[(b)] diverges at \(x=1\) but converges at \(x=-1\);
\item[(c)] converges at \(x=1\) but diverges at \(x=-1\);
\item[(d)] converges at both \(x=1\) and \(x=-1\);
\item[(e)] converges pointwise on \((-1,1)\), but does not converge uniformly on \((-1,1)\).
\end{enumerate}
\end{problem}

\begin{problem}[25pts \noindent\textbf{Exercise 4.2.7.} ]
    Let \(m \ge 0\) be a positive integer, and let \(0  < r\) be real numbers.  
Prove the identity
\[
\frac{r}{\,r-x\,} = \sum_{n=0}^{\infty} x^n r^{-n}
\]
for all \(x \in (-r, r)\).  

Using Proposition 4.2.6, conclude the identity
\[
\frac{r}{(r-x)^{\,m+1}}
= \sum_{n=m}^{\infty} \frac{n!}{m!(n-m)!}\, x^{\,n-m} r^{-n}
\]
for all integers \(m \ge 0\) and all \(x \in (-r, r)\).  
Also explain why the series on the right-hand side is absolutely convergent.
\end{problem}

\begin{problem}[25pts]
    Let \(E\) be a subset of \(\mathbb{R}\), let \(a\) be an interior point of \(E\), and let \(f:E\to\mathbb{R}\) be a function which is real analytic at \(a\) and has a power series expansion
\[
f(x)=\sum_{n=0}^{\infty} c_n (x-a)^n
\]
at \(a\) which converges on the interval \((a-r,\, a+r)\). Let \((b-s,\, b+s)\) be any subinterval of \((a-r,\, a+r)\) for some \(s>0\).

\begin{enumerate}
\item[(a)] Prove that \(|a-b| \le r-s\), so in particular \(|a-b| < r\).

\item[(b)] Show that for every \(0<\varepsilon<r\), there exists a \(C>0\) such that \(|c_n| \le C(r-\varepsilon)^{-n}\) for all integers \(n\ge 0\).  
\emph{(Hint: what do we know about the radius of convergence of the series \(\sum_{n=0}^{\infty} c_n(x-a)^n\)?)}

\item[(c)] Show that the numbers \(d_0,d_1,\ldots\), given by the formula
\[
d_m := \sum_{n=m}^{\infty} \frac{n!}{m!(n-m)!}(b-a)^{\,n-m} c_n \qquad \text{for all integers } m\ge 0,
\]
are well-defined, in the sense that the above series is absolutely convergent.  
\emph{(Hint: use (b) and the comparison test, Corollary 7.3.2, followed by Exercise 4.2.7.)}

\item[(d)] Show that for every \(0<\varepsilon<s\) there exists a \(C>0\) such that
\[
|d_m| \le C(s-\varepsilon)^{-m}
\]
for all integers \(m\ge 0\).  
\emph{(Hint: use the comparison test, and Exercise 4.2.7.)}



\item[(e)] Show that the power series \(\sum_{m=0}^{\infty} d_m (x-b)^m\) is absolutely convergent for \(x \in (b-s,\, b+s)\) and converges to \(f(x)\).  
(You may need Fubini’s theorem for infinite series, Theorem 8.2.2 of \emph{Analysis I}, as well as Exercise 4.2.5. One may also need to use a variant of the \(d_m\) in which the \(c_n\) are replaced by \(|c_n|\).)

Note. You can use Exercise 4.2.5. Let \(a, b\) be real numbers, and let \(n \ge 0\) be an integer. Prove the identity
\[
(x-a)^n = \sum_{m=0}^{n} \frac{n!}{m!(n-m)!}(b-a)^{\,n-m}(x-b)^m
\]
for any real number \(x\).

\item[(f)] Conclude that \(f\) is real analytic at \(b\), and thus analytic at every point in \((a-r,\, a+r)\).

\end{enumerate}
\end{problem}

\begin{problem}[25pts]
    \vphantom{text}
  \begin{enumerate}
  
  \item[(a)] If each \(a_n \ge 0\) and if \(\sum a_n\) diverges, show that \(\sum a_n x^n \to +\infty\) as \(x \to 1^{-}\).  
(Assume \(\sum a_n x^n\) converges for \(|x|<1\).)

\item[(b)] If each \(a_n \ge 0\) and if \(\lim_{x \to 1^{-}} \sum a_n x^n\) exists and equals \(A\), prove that \(\sum a_n\) converges and has sum \(A\).  


  \end{enumerate} 
\end{problem}