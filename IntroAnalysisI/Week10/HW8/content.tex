\begin{problem}[25pts]
    Give examples of a formal power series
\[
\sum_{n=0}^{\infty} c_n x^n
\]
centered at \(0\) with radius of convergence \(1\), which
\begin{enumerate}
\item[(a)] diverges at both \(x=1\) and \(x=-1\);
\item[(b)] diverges at \(x=1\) but converges at \(x=-1\);
\item[(c)] converges at \(x=1\) but diverges at \(x=-1\);
\item[(d)] converges at both \(x=1\) and \(x=-1\);
\item[(e)] converges pointwise on \((-1,1)\), but does not converge uniformly on \((-1,1)\).
\end{enumerate}
\end{problem}
\begin{proof}[(a)]
    Suppose \(\left( c_n \right)_{n=0}^{\infty} = \left( 1 \right)_{n=0}^{\infty}   \), then the radius of convergence \(R = \frac{1}{\limsup_{n \to \infty} 1 } = 1\). Also, when \(x = 1\), the power series is \(\sum_{n=0}^{\infty} 1 \), which diverges, while \(x = -1\), \(\sum_{n=0}^{\infty} (-1)^n \) also diverges since it oscillates.     
\end{proof}

\begin{proof}[(b)]
    Suppose \(\left( c_n \right)_{n=1}^{\infty} = \left( \frac{1}{n} \right)_{n=1}^{\infty}   \), then if the radius of convergence is \(R\), then 
    \[
        R = \frac{1}{\limsup_{n \to \infty} \left( \frac{1}{n} \right)^{\frac{1}{n}}  }.
    \] 
    Note that
    \[
        \lim_{n \to \infty} \left( \frac{1}{n} \right)^{\frac{1}{n}} = \lim_{n \to \infty} n^{-\frac{1}{n}} = 1  
    \] since suppose \(y_n = n^{-\frac{1}{n}} \), then \(\ln y_n = -\frac{1}{n} \ln n\) for \(y > 0\), and thus 
    \[
        \lim_{n \to \infty} \ln y_n = \lim_{n \to \infty} -\frac{\ln n}{n} = \lim_{n \to \infty} \frac{\frac{1}{n}}{1} = 0,
    \] which means \(\lim_{n \to \infty} y_n = e^0 = 1 \). Note that \(\sum_{n=1}^{\infty} \frac{1}{n} \) diverges by the \(p\)-series test and \(\sum_{n=1}^{\infty} (-1)^n \frac{1}{n} \) converges since \(\lim_{n \to \infty} \frac{1}{n} = 0 \) and \(\left( \frac{1}{n} \right)_{n=1}^{\infty}  \) decreasing so we can use alternating series test. Hence, the power series 
    \[
        \sum_{n=1}^{\infty} \frac{1}{n} x^n 
    \]
    diverges at \(x = 1\) but converges at \(x = -1\).       
\end{proof}

\begin{proof}[(c)]
    Suppose \(\left( c_n \right) = \left( (-1)^n \frac{1}{n} \right)_{n=1}^{\infty}   \), then the radius converges is 
    \[
        R = \frac{1}{\limsup_{n \to \infty} \left\vert (-1)^n \frac{1}{n} \right\vert^{\frac{1}{n}}  } = \frac{1}{\limsup_{n \to \infty} \left( \frac{1}{n} \right)^{\frac{1}{n}}  } = 1
    \] by (b). Now if \(x = 1\), then the series becomes 
    \[
        \sum_{n=1}^{\infty} (-1)^n \frac{1}{n}, 
    \] which has been argued to be convergent in (b). If \(x = -1\), then the series becomes 
    \[
        \sum_{n=1}^{\infty} (-1)^n \frac{1}{n} (-1)^n = \sum_{n=1}^{\infty} \frac{1}{n},  
    \] which has been argued to be convergent in (b). Hence, this power series converges at \(x=1\) but diverges at \(x = -1\).   
\end{proof}

\begin{proof}[(d)]
    Suppose \(\left( c_n \right)_{n=1}^{\infty} = \left( \frac{1}{n^2} \right)_{n=1}^{\infty} \), then the radius of convergence is 
    \[
        R = \frac{1}{\limsup_{n \to \infty} \left( \frac{1}{n^2} \right)^{\frac{1}{n}}  } = 1
    \] since \(\lim_{n \to \infty} \left( \frac{1}{n^2} \right)^{\frac{1}{n}} = 1  \). Now if \(x = 1\), then the series becomes 
    \[
        \sum_{n=1}^{\infty} \frac{1}{n^2} 
    \] converges by \(p\)-series test, while if \(x = -1\), then the series becomes 
    \[
        \sum_{n=1}^{\infty} (-1)^n \frac{1}{n^2}, 
    \] which converges by alternating series test. Hence, this power series converges at both \(x = 1\) and \(x = -1\).  
\end{proof}

\begin{proof}[(e)]
    Consider the power series 
    \[
        \sum_{n=0}^{\infty} x^n = \frac{1}{1-x},
    \] then we know the radius of convergence of this power series is \(1\). Now we show that it converges pointwise on \((-1,1)\) but not uniformly on \((-1,1)\). First, suppose
    \[
        S_N(x) = \sum_{n=0}^N x^n = \frac{1 - x^{N+1}}{1-x}, 
    \]   
    then for all \(\varepsilon > 0\) and fixed \(x \in (-1, 1)\), there exists \(N > 0\) s.t. \(x^{N+1} < (1-x) \varepsilon \), so for all \(n \ge N\) we have 
    \[
        \left\vert \frac{1 - x^{n+1}}{1-x} - \frac{1}{1-x} \right\vert = \frac{x^{n+1}}{1-x} \le \frac{x^{N+1}}{1-x} < \varepsilon, 
    \] 
    so it converges pointwise on \((-1,1)\). Now we show that it does not converge uniformly on \((-1, 1)\). If so, then for some fixed \(N \in \mathbb{N} \) we have 
    \[
        \frac{x^{N+1}}{1-x} < \varepsilon, \quad \forall x \in (-1, 1) \text{ and } \varepsilon > 0,
    \] but 
    \[
        \lim_{x \to 1^-} \frac{x^{N+1}}{1-x} = \infty, 
    \]    
    so this is impossible, and thus \(\sum_{n=0}^{\infty} x^n \) does not converge uniformly on \((-1, 1)\).     
\end{proof}

\begin{problem}[25pts \noindent\textbf{Exercise 4.2.7.}]
    Let \(m \ge 0\) be a positive integer, and let \(0  < r\) be real numbers.  
Prove the identity
\[
\frac{r}{\,r-x\,} = \sum_{n=0}^{\infty} x^n r^{-n}
\]
for all \(x \in (-r, r)\).  

Using Proposition 4.2.6, conclude the identity
\[
\frac{r}{(r-x)^{\,m+1}}
= \sum_{n=m}^{\infty} \frac{n!}{m!(n-m)!}\, x^{\,n-m} r^{-n}
\]
for all integers \(m \ge 0\) and all \(x \in (-r, r)\).  
Also explain why the series on the right-hand side is absolutely convergent.
\end{problem}
\begin{proof}
    Fix \(x \in (-r, r)\). Suppose 
    \[
        S_N(x) = \sum_{n=0}^N x^{n} r^{-n} = \sum_{n=0}^N \left( \frac{x}{r} \right)^{n} = \frac{1 - \left( \frac{x}{r} \right)^{N+1} }{1 - \frac{x}{r}},  
    \] then we know 
    \[
        \lim_{N \to \infty} S_N(x) = \lim_{N \to \infty} \frac{1 - \left( \frac{x}{r} \right)^{N+1} }{1 - \frac{x}{r}} = \frac{1}{1 - \frac{x}{r}} = \frac{r}{r - x}  
    \] since \(\left\vert \frac{x}{r} \right\vert  < 1\). Hence, 
    \[
        \frac{r}{r-x} = \sum_{n=0}^{\infty} x^n r^{-n}, \quad \forall x \in (-r, r). 
    \]
    Now we show 
    \[
        \frac{r}{(r-x)^{\,m+1}} = \sum_{n=m}^{\infty} \frac{n!}{m!(n-m)!}\, x^{\,n-m} r^{-n}
    \] for all integers \(m \ge 0\) and all \(x \in (-r, r)\) by induction. 
    \begin{itemize}
        \item Base case: for \(m = 0\), we have proved that it is true. 
        \item Now suppose for \(m = k-1\) this is true, then 
        \[
            \frac{r}{(r-x)^k} = \sum_{n=k-1}^{\infty} \frac{n!}{(k-1)!(n-k+1)!}x^{n-k+1} r^{-n}, 
        \] then by Proposition 4.2.6 we know it is differentiable for all \(x \in (-r, r)\) and thus 
        \[
            k \cdot r (r-x)^{-k-1} = \sum_{n=k}^{\infty} \frac{n!}{(k-1)!(n-k+1)!} \cdot (n-k+1) x^{n-k} r^{-n}, 
        \] which gives 
        \[
            \frac{r}{(r-x)^{k+1}} = \sum_{n=k}^{\infty} \frac{n!}{k!(n-k)!} x^{n-k} r^{-n}. 
        \]
        Hence, this statement is true for \(m = k\) and all \(x \in (-r, r)\), and thus the induction is finished. Now we show 
        \[
            \sum_{n=m}^{\infty} \frac{n!}{m!(n-m)!}x^{n-m}r^{-n} 
        \] converges absolutely for all \(x \in (-r, r)\). We can first rewrite the series by letting \(k = n-m\), so
        \[
            \sum_{n=m}^{\infty} \frac{n!}{m!(n-m)!}x^{n-m}r^{-n} = \sum_{k=0}^{\infty} \frac{(k+m)!}{m!k!}x^k r^{-k-m}. 
        \]  
        Hence, we know 
        \begin{align*}
            \lim_{k \to \infty} \left\vert \frac{\frac{(k+m+1)!}{m!(k+1)!} x^{k+1} r^{-k-m-1}}{\frac{(k+m)!}{m!k!} x^k r^{-k-m}} \right\vert = \lim_{k \to \infty} \left\vert \frac{(k+m+1)x}{(k+1)r} \right\vert = \left\vert \frac{x}{r} \right\vert < 1,    
        \end{align*}
        and thus by ratio test we know this power series converges absolutely.
    \end{itemize}  
\end{proof}

\begin{problem}[25pts]
    Let \(E\) be a subset of \(\mathbb{R}\), let \(a\) be an interior point of \(E\), and let \(f:E\to\mathbb{R}\) be a function which is real analytic at \(a\) and has a power series expansion
\[
f(x)=\sum_{n=0}^{\infty} c_n (x-a)^n
\]
at \(a\) which converges on the interval \((a-r,\, a+r)\). Let \((b-s,\, b+s)\) be any subinterval of \((a-r,\, a+r)\) for some \(s>0\).

\begin{enumerate}
\item[(a)] Prove that \(|a-b| \le r-s\), so in particular \(|a-b| < r\).

\item[(b)] Show that for every \(0<\varepsilon<r\), there exists a \(C>0\) such that \(|c_n| \le C(r-\varepsilon)^{-n}\) for all integers \(n\ge 0\).  
\emph{(Hint: what do we know about the radius of convergence of the series \(\sum_{n=0}^{\infty} c_n(x-a)^n\)?)}

\item[(c)] Show that the numbers \(d_0,d_1,\ldots\), given by the formula
\[
d_m := \sum_{n=m}^{\infty} \frac{n!}{m!(n-m)!}(b-a)^{\,n-m} c_n \qquad \text{for all integers } m\ge 0,
\]
are well-defined, in the sense that the above series is absolutely convergent.  
\emph{(Hint: use (b) and the comparison test, Corollary 7.3.2, followed by Exercise 4.2.7.)}

\item[(d)] Show that for every \(0<\varepsilon<s\) there exists a \(C>0\) such that
\[
|d_m| \le C(s-\varepsilon)^{-m}
\]
for all integers \(m\ge 0\).  
\emph{(Hint: use the comparison test, and Exercise 4.2.7.)}



\item[(e)] Show that the power series \(\sum_{m=0}^{\infty} d_m (x-b)^m\) is absolutely convergent for \(x \in (b-s,\, b+s)\) and converges to \(f(x)\).  
(You may need Fubini’s theorem for infinite series, Theorem 8.2.2 of \emph{Analysis I}, as well as Exercise 4.2.5. One may also need to use a variant of the \(d_m\) in which the \(c_n\) are replaced by \(|c_n|\).)

Note. You can use Exercise 4.2.5. Let \(a, b\) be real numbers, and let \(n \ge 0\) be an integer. Prove the identity
\[
(x-a)^n = \sum_{m=0}^{n} \frac{n!}{m!(n-m)!}(b-a)^{\,n-m}(x-b)^m
\]
for any real number \(x\).

\item[(f)] Conclude that \(f\) is real analytic at \(b\), and thus analytic at every point in \((a-r,\, a+r)\).

\end{enumerate}
\end{problem}

\begin{proof}[(a)]
    $(b-s, b+s) \subseteq (a-r, a+r)$ means 
    \[
        \begin{cases}
        b - s \ge a - r, \\
        b + s \le a + r, \\
        r \ge s.
        \end{cases}
    \]
    This implies that $\lvert a-b \rvert \le r-s$, and since $s > 0$, $\lvert a-b \rvert < r$.
\end{proof}

\begin{proof}[(b)]
    The series converges for every $x$ with $\lvert x-a \rvert < r$. For fixed $\varepsilon > 0$, we can pick $x_0$ such that $x_0-a = r - \frac{\varepsilon}{2} < r$. Then the series $\sum_{n=0}^{\infty} c_n(x_0-a)^n$ converges absolutely, i.e.
    \[
    \sum_{n=0}^{\infty} \lvert c_n \rvert \lvert (x_0-a) \rvert^n  < \infty
    \]
    Let $M = \sum_{n=0}^{\infty} \lvert c_n \rvert \lvert (x_0-a) \rvert^n$. Since
    \[
    \forall n, \lvert c_n \rvert \lvert (x_0-a) \rvert^n \ge 0, \text{so }\lvert c_n \rvert \lvert (x_0-a) \rvert^n \le M \text{ for all $n$}.
    \]
    And this implies 
    \[
    \forall n, \lvert c_n \rvert \le M \lvert (x_0-a) \rvert^{-n}.
    \]
    Since $x_0-a = r - \frac{\varepsilon}{2} > 0$, $\lvert c_n \rvert \le M (r-\frac{\varepsilon}{2})^{-n}$, and $r - \frac{\varepsilon}{2} > r - \varepsilon$, so $\lvert c_n \rvert \le M (r-\varepsilon)^{-n}$, and hence 
    \[
    \forall \varepsilon, \exists C = M \text{ such that } \lvert c_n \rvert \le M (r-\varepsilon)^{-n}.
    \]
    
\end{proof}

\begin{proof}[(c)]
    From (a) we have $\lvert a-b \rvert < r$, we can pick $\varepsilon > 0$ such that $0 < \varepsilon < r - \lvert b-a \rvert$, and let $q = \frac{\lvert b-a \rvert}{r-\varepsilon} < 1$. \\
    By (b), $\exists C > 0$ with $\lvert c_n \rvert \le C (r-\varepsilon)^{-n}$ for all $n$. Then $\forall n \ge m$,
    \[
    \lvert \binom{n}{m} (b-a)^{n-m} c_n \rvert \le C \binom{n}{m} \lvert b-a \rvert^{n-m} (r-\varepsilon)^{-n} = C\binom{n}{m}q^{n-m} (r-\varepsilon)^{-m}
    \]
    \begin{claim}
        $\sum_{k=0}^{\infty} \binom{m+k}{m} q^k = \frac{1}{(1-q)^{m+1}}$ respect to $\lvert q \rvert < 1$
    \end{claim}
    \begin{explanation}
        By geometric series property, we know 
        \[
        \frac{1}{1-q} = \sum_{k=0}^{\infty} q^k , \text{ if $\lvert q \rvert $ < 1}
        \]
        Then we differentiate both side for $m$ times, for left hand side, we can get:
        \[
        \frac{d^m}{dq^m} \frac{1}{1-q} = \frac{d^m}{dq^m} (1-q)^{-1} = (m)(m-1)...(1)(1-q)^{-(m+1)} = \frac{m!}{(1-q)^{m+1}}
        \]
        And for right hand side, we can get:
        \[
        \frac{d^m}{dq^m} \sum_{k=0}^{\infty} q^k = \sum_{k=m}^{\infty} \frac{k!}{(k-m)!} q^{k-m}
        \]
        And let $j = k-m$,
        \[
        \sum_{k=m}^{\infty} \frac{k!}{(k-m)!} q^{k-m} = \sum_{j=0}^{\infty} \frac{(j+m)!}{j!} q^{j} = m! \sum_{j=0}^{\infty} \binom{j+m}{m} q^{j}
        \]
        And compare both side, and cancel $m!$, we can conclude that
        \[
        \frac{1}{(1-q)^{m+1}} = \sum_{k=0}^{\infty} \binom{m+k}{m} q^{k}
        \]
    \end{explanation}
    Then we let $k = n-m$, we know
    \[
    \lvert \binom{n}{m} (b-a)^{n-m} c_n \rvert \le C (r-\varepsilon)^{-m}\binom{m+k}{m}q^{k}
    \]
    \[
    \sum_{k=0}^{\infty}C(r-\varepsilon)^{-m}\binom{m+k}{m}q^k = C(r-\varepsilon)^{-m}\sum_{k=0}^{\infty}\binom{m+k}{m}q^k = C(r-\varepsilon)^{-m}\frac{1}{(1-q)^{m+1}}
    \]
    ,which is bounded. So
    \[
    d_m \coloneqq \sum_{n=m}^{\infty} \lvert \binom{n}{m} (b-a)^{n-m} c_n \rvert \le  \sum_{k=0}^{\infty}C(r-\varepsilon)^{-m}\binom{m+k}{m}q^k = \frac{C(r-\varepsilon)^{-m}}{(1-q)^{m+1}} < \infty
    \]
    Hence the series $d_m$ is absolutely convergent $\implies$ $d_m$ is welled-defined.
\end{proof}

\begin{proof}[(d)]
    From (a), we have $\lvert b-a \rvert < r-s$. \\
    Fixed $0 < \varepsilon < s \le r$, let $\delta = \frac{\varepsilon}{2}$ and $R = r - \delta$ ($\lvert b-a \rvert < R < r$).\\
    By part (b), 
    \[
    \forall n, \exists C_0 \text{ such that } \lvert c_n \rvert \le C_0R^{-n},
    \]
    and also $R > \lvert b-a \rvert \implies \frac{\lvert b-a \rvert }{R} < 1$.
    \[
    \begin{aligned}
    \lvert d_m \rvert &\le  \sum_{n=m}^{\infty}\binom{n}{m} \lvert (b-a) \rvert^{n-m} \lvert c_n \rvert \le C_0 \sum_{n=m}^{\infty}\binom{n}{m} \lvert (b-a) \rvert^{n-m} R^{-n} \\
    &= C_0\sum_{n=m}^{\infty}\binom{n}{m} (\frac{\lvert b-a \rvert}{R})^{n-m} R^{-m} = C_0R^{-m}\sum_{k=0}^{\infty}\binom{m+k}{m} \lvert (\frac{\lvert b-a \rvert}{R}) \rvert^k \\
    &= C_0R^{-m}\frac{1}{(1-\frac{\lvert b-a \rvert}{R})^{m+1}} = C_0\frac{R}{(R-\lvert b-a \rvert)}\frac{1}{(R-\lvert b-a \rvert)^m},
    \end{aligned}
    \]
    and 
    \[R - \lvert b-a \rvert > (r - \frac{\varepsilon}{2}) - (r-s) = s - \frac{\varepsilon}{2} > s - \varepsilon
    \]
    This will imply 
    \[
    (R - \lvert b-a \rvert)^{-m} < (s - \varepsilon)^{-m} \text{ , and  }\frac{R}{(R-\lvert b-a \rvert)} = \frac{r - \frac{\varepsilon}{2}}{r - \frac{\varepsilon}{2} - \lvert b-a \rvert} \le \frac{r - \frac{\varepsilon}{2}}{s - \frac{\varepsilon}{2}}
    \]
    Therefore, when $C \coloneqq C_0 \cdot\frac{r - \frac{\varepsilon}{2}}{s - \frac{\varepsilon}{2}}$, $\lvert d_m \rvert \le C (s-\varepsilon)^{-m}, \forall m \ge 0$.
\end{proof}

\begin{proof}[(e)]

Fix with $\lvert x-b \rvert < s$, we can choose $0 < \varepsilon < s - \lvert x-b \rvert$. \\
From (d),
\[
\forall m \ge 0, \exists C \text{ such that } \lvert d_m \rvert \le C(s - \varepsilon)^{-m},
\]
and hence
\[
\sum_{m=0}^{\infty} \lvert d_m \rvert \lvert x-b \rvert^m \le C \sum_{m=0}^{\infty} (\frac{\lvert x-b\rvert}{s- \varepsilon})^m < \infty
\]
So $\sum_{m=0}^{\infty}d_m(x-b)^m$ is absolutely convergent for $\lvert x-b \rvert < s$. \\

The binomial identity tell us
\[
(x-a)^n = \sum_{m=0}^{n} \binom{n}{m}(b-a)^{n-m}(x-b)^m
\]
So
\[
\sum_{n=0}^{\infty}(\lvert b-a\rvert + \lvert x-b \rvert)^n \lvert c_n \rvert = \sum_{n=0}^{\infty}\sum_{m=0}^{n} \binom{n}{m} \lvert b-a\rvert^{n-m}\lvert x-b \rvert^m \lvert c_n \rvert
\]
Since $\lvert b-a \rvert \le r-s$ and $\lvert x-b \rvert < s$, $\lvert b-a\rvert + \lvert x-b \rvert < r$, we can pick $\delta$ that $0 < \delta < r - (\lvert b-a\rvert + \lvert x-b \rvert)$. \\
By (b), $\exists C_0$ such that $\lvert c_n \rvert \le C_0(r-\delta)^{-n}$. Therefore
\[
\sum_{n=0}^{\infty}(\lvert b-a\rvert + \lvert x-b \rvert)^n \lvert c_n \rvert \le C_0\sum_{n=0}^{\infty}(\frac{\lvert b-a\rvert + \lvert x-b \rvert}{r-\delta})^n < \infty
\]
Thus the double series of absolute values converge, and Fubini's theorem justifies
\[
\sum_{n=0}^{\infty}c_n\sum_{m=0}^{n} \binom{n}{m} (b-a)^{n-m}(x-b)^m = \sum_{m=0}^{\infty}(\sum_{n=m}^{\infty} \binom{n}{m} (b-a)^{n-m}c_n)(x-b)^m
\]
And this imply
\[
\sum_{n=0}^{\infty}c_n(x-a)^n = \sum_{m=0}^{\infty} d_m(x-b)^m
\]
Since the left hand side equals $f(x)$ for $x\in (a-r, a+r)$, and right hand side is absolutely convergent for $x \in (b-s, b+s)$, and from (a) we know $(b-s, b+s) \subseteq (a-r, a+r)$, so 
\[
\forall x \in (b-s, b+s), \sum_{m=0}^{\infty} d_m(x-b)^m \text{ converges absolutely and equals to $f(x)$}.
\]

\end{proof}

\begin{proof}[(f)]
    Pick any $b \in (a-r, a+r)$, then $b - (a-r) > 0$ and $(a+r) - b > 0$, then we can choose
    \[
    s = \frac{1}{48763} \min\{b-(a-r), (a+r) -b\} > 0, \text{ so that } (b-s, b+s) \subseteq (a-r, a+r)
    \]
    Part (e) we proved the series $\sum_{m=0}^{\infty} d_m(x-b)^m$ is absolutely convergent for all $x \in (b-s, b+s)$, and equals to $f(x)$. This means $f$ has power-series expansion centered at $b$ that converges to $f$ in a neighborhood of $b$, and hence $f$ is real-analytic at $b$.\\
    Since $b$ is arbitrary in $(a-r, a+r)$, we can use same argument for all $b \in (a-r, a+r)$, so $f$ is real-analytic at every point of $(a-r, a+r)$.
\end{proof}

\begin{problem}[25pts]
    \vphantom{text}
  \begin{enumerate}
  
  \item[(a)] If each \(a_n \ge 0\) and if \(\sum a_n\) diverges, show that \(\sum a_n x^n \to +\infty\) as \(x \to 1^{-}\).  
(Assume \(\sum a_n x^n\) converges for \(|x|<1\).)

\item[(b)] If each \(a_n \ge 0\) and if \(\lim_{x \to 1^{-}} \sum a_n x^n\) exists and equals \(A\), prove that \(\sum a_n\) converges and has sum \(A\).  


  \end{enumerate} 
\end{problem}

\begin{proof}[(a)]
    Let $f(x) = \sum_{n=0}^{\infty} a_nx^n$, which converges for $\lvert x \rvert < 1$.\\
    % Since all $a_n \ge 0$, the partial sums 
    % \[
    % S_N(x) = \sum_{n=0}^{N} a_nx^n 
    % \]
    % increase in both $N$ and $x$.\\
    Fix $M$, for $x < 1$,
    \[
    f(x) = \sum_{n=0}^{\infty} a_nx^n \ge \sum_{n=0}^{M} a_nx^n
    \]
    Now let $x \to 1^-$,
    \[
    \lim_{x \to 1^-} f(x) \ge \lim_{x \to 1^-} \sum_{n=0}^{M} a_nx^n = \sum_{n=0}^{M} a_n
    \]
    Then we show that $\sum a_n \to \infty$. Suppose it is not, then $\sum a_n$ is bounded, and we know for all n, $a_n \ge 0$. this means $\sum a_n$ is monotonically increasing, and hence $\sum a_n$ converges, which is a contradiction. \\
    Since $\sum a_n \to \infty$, we can make $\sum_{n=0}^{M} a_n$ arbitrarily large by choosing $M$ large enough, and hence $f(x) \to + \infty$ as $x \to 1^-$.

\end{proof}

\begin{proof}[(b)]
    Let $S_N = \sum_{n=0}^{N} a_n$ \\
    Let $f(x) = \sum_{n=0}^{\infty} a_nx^n \to A < \infty$ as $x \to 1^-$. \\
    Since $f(x)$ converges at $x = 1^-$, the radius of convergence $R \ge 1$. \\
    Suppose $S_N$ didn't converge; then $S_N = \sum a_n$ diverges. by (a), $\sum a_n x^n$ diverges, and hence $f(x) \to \infty$, which is a contradiction. Therefore, $S_N$ must converge, and hence $\sum a_n$ converges. \\
    Suppose $\sum a_n$ converges to $S$. We separate to two cases depend on the value of $R$:
    \begin{itemize}
        \item Case 1$(R > 1)$: $f(x) = \sum_{n=0}^{\infty} a_nx^n$ is converge at $x = 1$, 
        \[
            A = \lim_{x \to 1^-} f(x) = f(1) = \sum_{n=0}^{\infty} a_n1^n = \sum_{n=0}^{\infty} a_n = S.
        \]
        \item Case 2$(R = 1)$: Since $\sum a_n$ converge, Abel's theorem tells us that
        \[
            A = \lim_{x \to 1^-} f(x) = \lim_{x \to 1^-} \sum_{n=0}^{\infty} a_nx^n = \sum_{n=0}^{\infty} a_n1^n = \sum_{n=0}^{\infty} a_n =S,
        \]
    \end{itemize}
    
    and hence $A = S$, so $\sum a_n$ converges to $A$.
\end{proof}