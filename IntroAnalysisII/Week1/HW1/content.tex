\begin{problem}
    Define $f : \mathbf{R}^2 \to \mathbf{R}$ by
$$
f(x,y)=
\begin{cases}
\displaystyle
\frac{x^2 y}{x^2+y^2}, & (x,y)\neq(0,0),\\[1em]
0, & (x,y)=(0,0).
\end{cases}
$$

\begin{itemize}
\item[(a)] Show that for every fixed direction $v\in\mathbf{R}^2$, 
the limit
$$
\lim_{t\to 0} \frac{f(tv)-f(0)}{t}
$$
exists.

\item[(b)] Show that $f$ is \emph{not} differentiable at $(0,0)$
in the sense of Definition 6.2.2.

\item[(c)] Explain precisely which part of the definition of differentiability fails.
\end{itemize}
\end{problem}

\begin{problem}
    Let $f:\mathbf{R}^n\to\mathbf{R}^m$ and suppose that for some linear map
$L:\mathbf{R}^n\to\mathbf{R}^m$ one has
$$
f(x_0+h)=f(x_0)+L(h)+R(h),
$$
where the remainder satisfies
$$
\|R(h)\| \le C \|h\|^{1+\alpha}
$$
for some constants $C>0$ and $\alpha>0$.

\begin{itemize}
\item[(a)] Prove that $f$ is differentiable at $x_0$ with derivative $L$.

\item[(b)] Show that if $\alpha=0$, the conclusion may fail
by constructing a counterexample.
\end{itemize}
\end{problem}

\begin{problem}
    Let $E=\{(x,y)\in\mathbf{R}^2: y\ge 0\}$, and define
$$
f(x,y)=0 \quad \text{for all } (x,y)\in E.
$$

\begin{itemize}
\item[(a)] Show that $f$ is differentiable at $(0,0)$
in the sense of Definition 6.2.2.

\item[(b)] Show that every linear map
$L:\mathbf{R}^2\to\mathbf{R}$
satisfying $L(x,0)=0$ for all $x$
also satisfies the differentiability condition.

\item[(c)] Conclude that the derivative at $(0,0)$
is not uniquely determined.
\end{itemize}

\end{problem}


\begin{problem}
    Let $E \subset \mathbf{R}^n$, let $f : E \to \mathbf{R}^m$, let $x_0$ be an interior point of $E$, and let $1 \le j \le n$.

Show that $\frac{\partial f}{\partial x_j}(x_0)$ exists if and only if
$D_{e_j}f(x_0)$ and $D_{-e_j}f(x_0)$ exist and are negatives of each other.
In this case,
$$
\frac{\partial f}{\partial x_j}(x_0)
=
D_{e_j}f(x_0).
$$
\end{problem}

\begin{problem}
    \textbf{Exercise 6.3.3.}
Let $f : \mathbf{R}^2 \to \mathbf{R}$ be defined by
$$
f(x,y)
=
\frac{x^3}{x^2 + y^2}
\quad \text{for } (x,y) \ne (0,0),
\qquad
f(0,0) = 0.
$$

Show that $f$ is not differentiable at $(0,0)$, even though it is
directionally differentiable in every direction at $(0,0)$.
Explain why this does not contradict Theorem 6.3.8.

\end{problem}

\begin{problem}
    Let $E\subset \mathbb{R}^n$, let $f:E\to \mathbb{R}^m$, and let $x_0$ be an interior point of $E$.

Assume that there exists a linear map $A:\mathbb{R}^n\to \mathbb{R}^m$
such that for every unit vector $v\in S^{n-1}$,
\[
\lim_{t\to 0^+}
\frac{f(x_0+t v)-f(x_0)}{t}
=
A(v).
\]

Suppose moreover that the above convergence is \emph{uniform in $v$ on the unit sphere $S^{n-1}$}, meaning that

\[
\lim_{t\to 0^+}
\sup_{v\in S^{n-1}}
\left\|
\frac{f(x_0+t v)-f(x_0)}{t}
-
A(v)
\right\|
=
0.
\]

Equivalently, for every $\varepsilon>0$ there exists $\delta>0$
such that for all $0<t<\delta$ and all $v\in S^{n-1}$,
\[
\left\|
\frac{f(x_0+t v)-f(x_0)}{t}
-
A(v)
\right\|
<
\varepsilon.
\]

Prove that $f$ is differentiable at $x_0$ and that $f'(x_0)=A$.

\medskip
\noindent\textbf{Hint.}
For $h\neq 0$, write
\[
h=\|h\|\,v
\quad \text{with } v=\frac{h}{\|h\|}\in S^{n-1}.
\]
Use linearity of $A$ to rewrite
\[
A(h)=\|h\|A(v),
\]
and compare
\[
\frac{f(x_0+h)-f(x_0)-A(h)}{\|h\|}
\]
with
\[
\frac{f(x_0+\|h\|v)-f(x_0)}{\|h\|}-A(v).
\]
Then apply the assumed uniform convergence.
\end{problem}