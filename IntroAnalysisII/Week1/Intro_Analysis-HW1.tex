\documentclass[addpoints]{exam}
% leqno 將方程式編號放在左側

%\printanswers

\usepackage[top=2.5cm,bottom=2cm,left=2.5cm,right=2.5cm,headsep=10pt,a4paper]{geometry} % 引用頁面幾何套件,設定頁面邊界

\usepackage{siunitx} % 角度
% \input{NewCommands}

\usepackage{amsmath,amssymb,amsthm} % 引用常見的數學符號與設定


\theoremstyle{definition}
\newtheorem*{definition}{Definition}
\newtheorem*{theorem}{Theorem}
\newtheorem*{remark}{Remark}

\usepackage{lastpage} % 取得最後頁碼




\usepackage{graphicx} % 引用圖檔內嵌入套件
\graphicspath{{Pictures/}} % 設定圖檔目錄
\usepackage{xcolor} %引用顏色套件
\usepackage{enumerate}
\usepackage{float}
\usepackage{hyperref}
\usepackage{mdwlist} % use suspend & resume to successive enumerate

\usepackage{CJKutf8} % 設定中文

\def \lflr{\left\lfloor} %自定義 floor function
\def \rflr{\right\rfloor} %自定義

% question separation
\renewcommand{\questionshook}{%
  \setlength{\itemsep}{0.5cm}%
}

\allowdisplaybreaks %equation allowed to next page

\firstpageheader{\today}
                {}
                {}
\footer{}{\sf Page \thepage~of~\pageref{LastPage}}{}

\begin{document}

\begin{CJK*}{UTF8}{gbsn}
% Title
\begin{center}
    {\Large{\bf Introduction to Mathematical Analysis II\\
    Homework 1  Due  March 6  (Friday), 2026\\
    Please submit your group homework online in PDF format.\\
You are expected to work collaboratively within your group. 
Please do not obtain complete solutions directly from AI tools. 
The purpose of this assignment is to develop your own mathematical reasoning and problem-solving skills. %  Homework X Brief Solution
  }} \\ 
    \large
    %\text{Due date: 3/9/2023}
\end{center}

\noindent\rule{16.2cm}{0.4pt}


\begin{questions}


% ============ Question x ============
\question  (20 pts)  

 Define $f : \mathbf{R}^2 \to \mathbf{R}$ by
$$
f(x,y)=
\begin{cases}
\displaystyle
\frac{x^2 y}{x^2+y^2}, & (x,y)\neq(0,0),\\[1em]
0, & (x,y)=(0,0).
\end{cases}
$$

\begin{itemize}
\item[(a)] Show that for every fixed direction $v\in\mathbf{R}^2$, 
the limit
$$
\lim_{t\to 0} \frac{f(tv)-f(0)}{t}
$$
exists.

\item[(b)] Show that $f$ is \emph{not} differentiable at $(0,0)$
in the sense of Definition 6.2.2.

\item[(c)] Explain precisely which part of the definition of differentiability fails.
\end{itemize}

\begin{solution} 
\begin{enumerate}
\item[(a)] 



\bigskip
\item[(b)] 


\bigskip
\item[(c)] 

\end{enumerate}

 \end{solution}

\question (15 pts) 
Let $f:\mathbf{R}^n\to\mathbf{R}^m$ and suppose that for some linear map
$L:\mathbf{R}^n\to\mathbf{R}^m$ one has
$$
f(x_0+h)=f(x_0)+L(h)+R(h),
$$
where the remainder satisfies
$$
\|R(h)\| \le C \|h\|^{1+\alpha}
$$
for some constants $C>0$ and $\alpha>0$.

\begin{itemize}
\item[(a)] Prove that $f$ is differentiable at $x_0$ with derivative $L$.

\item[(b)] Show that if $\alpha=0$, the conclusion may fail
by constructing a counterexample.
\end{itemize}


\begin{solution}

\begin{enumerate}
\item[(a)] 



\bigskip
\item[(b)] 




\end{enumerate} 
\end{solution}

\question (15 pts) 	
Let $E=\{(x,y)\in\mathbf{R}^2: y\ge 0\}$, and define
$$
f(x,y)=0 \quad \text{for all } (x,y)\in E.
$$

\begin{itemize}
\item[(a)] Show that $f$ is differentiable at $(0,0)$
in the sense of Definition 6.2.2.

\item[(b)] Show that every linear map
$L:\mathbf{R}^2\to\mathbf{R}$
satisfying $L(x,0)=0$ for all $x$
also satisfies the differentiability condition.

\item[(c)] Conclude that the derivative at $(0,0)$
is not uniquely determined.
\end{itemize}


\begin{solution} 
 
 \end{solution}

\question (15 pts) \textbf{Exercise 6.3.2.}
Let $E \subset \mathbf{R}^n$, let $f : E \to \mathbf{R}^m$, let $x_0$ be an interior point of $E$, and let $1 \le j \le n$.

Show that $\frac{\partial f}{\partial x_j}(x_0)$ exists if and only if
$D_{e_j}f(x_0)$ and $D_{-e_j}f(x_0)$ exist and are negatives of each other.
In this case,
$$
\frac{\partial f}{\partial x_j}(x_0)
=
D_{e_j}f(x_0).
$$


\begin{solution} 

\end{solution}

\question (15 pts)
\textbf{Exercise 6.3.3.}
Let $f : \mathbf{R}^2 \to \mathbf{R}$ be defined by
$$
f(x,y)
=
\frac{x^3}{x^2 + y^2}
\quad \text{for } (x,y) \ne (0,0),
\qquad
f(0,0) = 0.
$$

Show that $f$ is not differentiable at $(0,0)$, even though it is
directionally differentiable in every direction at $(0,0)$.
Explain why this does not contradict Theorem 6.3.8.

 

\begin{solution} 

\end{solution}

\question (20 pts) 
Let $E\subset \mathbb{R}^n$, let $f:E\to \mathbb{R}^m$, and let $x_0$ be an interior point of $E$.

Assume that there exists a linear map $A:\mathbb{R}^n\to \mathbb{R}^m$
such that for every unit vector $v\in S^{n-1}$,
\[
\lim_{t\to 0^+}
\frac{f(x_0+t v)-f(x_0)}{t}
=
A(v).
\]

Suppose moreover that the above convergence is \emph{uniform in $v$ on the unit sphere $S^{n-1}$}, meaning that

\[
\lim_{t\to 0^+}
\sup_{v\in S^{n-1}}
\left\|
\frac{f(x_0+t v)-f(x_0)}{t}
-
A(v)
\right\|
=
0.
\]

Equivalently, for every $\varepsilon>0$ there exists $\delta>0$
such that for all $0<t<\delta$ and all $v\in S^{n-1}$,
\[
\left\|
\frac{f(x_0+t v)-f(x_0)}{t}
-
A(v)
\right\|
<
\varepsilon.
\]

Prove that $f$ is differentiable at $x_0$ and that $f'(x_0)=A$.

\medskip
\noindent\textbf{Hint.}
For $h\neq 0$, write
\[
h=\|h\|\,v
\quad \text{with } v=\frac{h}{\|h\|}\in S^{n-1}.
\]
Use linearity of $A$ to rewrite
\[
A(h)=\|h\|A(v),
\]
and compare
\[
\frac{f(x_0+h)-f(x_0)-A(h)}{\|h\|}
\]
with
\[
\frac{f(x_0+\|h\|v)-f(x_0)}{\|h\|}-A(v).
\]
Then apply the assumed uniform convergence.
\begin{solution} 
\end{solution}

\end{questions}

You can do the following problems to practice. You don't have to submit the following problems.

\begin{questions}

\question   (Exercise 6.3.4.)
Let $f : \mathbf{R}^n \to \mathbf{R}^m$ be differentiable and suppose that
$f'(x) = 0$ for all $x \in \mathbf{R}^n$.

\begin{itemize}
\item[(a)] Show that $f$ is constant on $\mathbf{R}^n$.
\item[(b)] For a greater challenge, replace $\mathbf{R}^n$ by an open connected subset
$\Omega \subset \mathbf{R}^n$ and prove the same result.
\end{itemize}


\begin{solution}

\end{solution}
\medskip

\question  Define $f:\mathbf{R}^2\to \mathbf{R}$ by
\[
f(x_1,x_2):=|x_1|.
\]
\begin{itemize}
\item[(a)] Show that for every $v=(v_1,v_2)\in \mathbf{R}^2$, the one-sided directional derivative
\[
D_v f(0,0)=\lim_{t\to 0^+}\frac{f(tv)-f(0,0)}{t}
\]
exists, and compute it explicitly.
\item[(b)] Show that the map $v\mapsto D_v f(0,0)$ is \emph{not} linear.
\item[(c)] Conclude that $f$ is not differentiable at $(0,0)$.
\end{itemize}
(\emph{Comment:} This illustrates precisely why Lemma 6.3.5 is only one-way: differentiability forces linear dependence on $v$, but existence of $D_v f(x_0)$ for all $v$ does not.)

\begin{solution}

\end{solution}

\medskip

\question  Define $f:\mathbf{R}^2\to \mathbf{R}$ by
\[
f(x,y)=
\begin{cases}
\dfrac{x^2y}{x^2+y^2}, & (x,y)\neq (0,0),\\[0.6em]
0, & (x,y)=(0,0).
\end{cases}
\]

\begin{itemize}
\item[(a)] Show that the partial derivatives $\dfrac{\partial f}{\partial x}(0,0)$ and
$\dfrac{\partial f}{\partial y}(0,0)$ both exist, and compute their values.

\item[(b)] Show that $f$ is \emph{not} differentiable at $(0,0)$.
(\emph{Hint:} Test the differentiability definition along a suitable curve such as $(x,y)=(t,t)$ or $(t,t^2)$ and compare the size of
$f(x,y)-f(0,0)-L(x,y)$ for any candidate linear map $L$.)

\item[(c)] Prove that at least one of the partial derivative functions
$\dfrac{\partial f}{\partial x}$ or $\dfrac{\partial f}{\partial y}$
fails to be continuous at $(0,0)$.
(\emph{Hint:} Compute $\dfrac{\partial f}{\partial x}(0,y)$ for $y\neq 0$ by restricting to the line $y=\text{const}$, and examine the limit as $y\to 0$.)
\end{itemize}

\medskip
\noindent\emph{This problem shows that the existence of partial derivatives at a point is not enough for differentiability; the continuity hypothesis in Theorem 6.3.8 is genuinely needed.}


\begin{solution}

\end{solution}

\end{questions}

\end{CJK*}
\end{document}

\question   

\begin{solution}
\end{solution}

\question 



\begin{solution}

\end{solution}

\question
\begin{solution}

\end{solution}





%
% \appendix
% \section{}


\question

\begin{solution}

\end{solution}
