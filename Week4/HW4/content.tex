\begin{problem}[16pts]
  \vphantom{text}
  \begin{enumerate}
  \item[(a)] Let
\[
X := \left\{ (a_n)_{n=0}^\infty : \sum_{n=0}^\infty |a_n| < \infty \right\}
\]
be the space of absolutely convergent sequences. Define the $\ell^1$ and $\ell^\infty$ metrics on this space by
\[
d_{\ell^1}\big((a_n)_{n=0}^\infty,(b_n)_{n=0}^\infty\big)
:= \sum_{n=0}^\infty |a_n - b_n|,
\]
\[
d_{\ell^\infty}\big((a_n)_{n=0}^\infty,(b_n)_{n=0}^\infty\big)
:= \sup_{n\in\mathbb{N}} |a_n - b_n|.
\]

Show that these are both metrics on $X$, but show that there exist sequences 
\[
x^{(1)}, x^{(2)}, \dots
\]
of elements of $X$ (i.e.\ sequences of sequences) which are convergent with respect to the $d_{\ell^\infty}$ metric but not with respect to the $d_{\ell^1}$ metric. Conversely, show that any sequence which converges in the $d_{\ell^1}$ metric automatically converges in the $d_{\ell^\infty}$ metric.


 \item[(b)] Let $(X,d_{\ell^1})$ be the metric space from part (a).  
For each natural number $n$, let $e^{(n)} = (e^{(n)}_j)_{j=0}^\infty$ be the sequence in $X$ such that  
\[
e^{(n)}_j := 
\begin{cases}
1, & \text{if } n=j,\\
0, & \text{if } n\neq j.
\end{cases}
\]

Show that the set
\[
\{ e^{(n)} : n \in \mathbb{N} \}
\]
is a closed and bounded subset of $X$, but is not compact.  

(This is despite the fact that $(X,d_{\ell^1})$ is even a complete metric space---a fact which we will not prove here.  
The problem is not that $X$ is incomplete, but rather that it is ``infinite-dimensional,'' in a sense that we will not discuss here.)

 
  \end{enumerate}
\end{problem}
%────────────────────────────────────────────────────────────────────────────────────────────────────────────────────────────────────────────────────

\begin{problem}[24pts]
  A metric space $(X,d)$ is called \emph{totally bounded} if for every $\varepsilon > 0$, there exists a natural number $n$ and a finite number of balls
\[
B(x^{(1)},\varepsilon), \; B(x^{(2)},\varepsilon), \; \dots, \; B(x^{(n)},\varepsilon)
\]
which cover $X$ (i.e.\ $X = \bigcup_{i=1}^n B(x^{(i)},\varepsilon)$).

\begin{enumerate}
\item[(a)] Show that every totally bounded space is bounded.

\item[(b)] Show the following stronger version of Proposition~1.5.5: if $(X,d)$ is compact, then it is complete and totally bounded.  
\emph{Hint:} if $X$ is not totally bounded, then there is some $\varepsilon > 0$ such that $X$ cannot be covered by finitely many $\varepsilon$-balls.  
Then use Exercise~8.5.20  (on page 182 of Analysis I)to find an infinite sequence of balls $B(x^{(n)},\varepsilon/2)$ which are disjoint from each other. Use this to construct a sequence which has no convergent subsequence.

\item[(c)] Conversely, show that if $X$ is complete and totally bounded, then $X$ is compact.  
\emph{Hint:} if $(x^{(n)})_{n=1}^\infty$ is a sequence in $X$, use the total boundedness hypothesis to recursively construct a sequence of subsequences $(x^{(n;j)})_{n=1}^\infty$ of $(x^{(n)})_{n=1}^\infty$ for each positive integer $j$, such that for each $j$ the elements of the sequence $(x^{(n;j)})_{n=1}^\infty$ are contained in a single ball of radius $1/j$.  
Also ensure that each sequence $(x^{(n;j+1)})_{n=1}^\infty$ is a subsequence of the previous one $(x^{(n;j)})_{n=1}^\infty$.  
Then show that the ``diagonal'' sequence $(x^{(n;n)})_{n=1}^\infty$ is a Cauchy sequence, and then use the completeness hypothesis.
\end{enumerate}
\end{problem}
%────────────────────────────────────────────────────────────────────────────────────────────────────────────────────────────────────────────────────

\begin{problem}[16pts]
  \vphantom{text}
  \begin{enumerate}
  \item[(a)]  A metric space $(X,d)$ is compact if and only if every sequence in $X$ has at least one limit point in $X$.

  
    \item[(b)] 
Let $(X,d)$ have the property that every open cover of $X$ has a finite subcover.  
Show that $X$ is compact.  

\emph{Hint:} If $X$ is not compact, then by part (a) there is a sequence $(x^{(n)})_{n=1}^\infty$ with no limit points.  
Then for every $x \in X$ there exists a ball $B(x,\varepsilon)$ containing $x$ which contains at most finitely many elements of this sequence.  
Now use the hypothesis.
  \end{enumerate}
\end{problem}
%────────────────────────────────────────────────────────────────────────────────────────────────────────────────────────────────────────────────────

\begin{problem}[10pts]
  Let $(X,d)$ be a compact metric space. Suppose that $(K_\alpha)_{\alpha \in I}$ is a collection of closed sets in $X$ with the property that any finite subcollection of these sets necessarily has non-empty intersection, thus
\[
\bigcap_{\alpha \in F} K_\alpha \neq \varnothing \quad \text{for all finite } F \subseteq I.
\]
(This property is known as the \emph{finite intersection property}.)  

Show that the entire collection has non-empty intersection, thus
\[
\bigcap_{\alpha \in I} K_\alpha \neq \varnothing.
\]

Show by counterexample that this statement fails if $X$ is not compact.
 
\end{problem}
%────────────────────────────────────────────────────────────────────────────────────────────────────────────────────────────────────────────────────

\begin{problem}[24pts]
  \vphantom{text}
  \begin{enumerate}

  \item[(a)] 
Let $(X,d)$ be a metric space, and let $(E,d|_{E \times E})$ be a subspace of $(X,d)$.  
Let $\iota_{E \to X} : E \to X$ be the inclusion map, defined by setting 
\[
\iota_{E \to X}(x) := x \quad \text{for all } x \in E.
\]  
Show that $\iota_{E \to X}$ is continuous.

 \item[(b)] Let $f : X \to Y$ be a function from one metric space $(X,d_X)$ to another $(Y,d_Y)$.  
Let $E$ be a subset of $X$ (which we give the induced metric $d_X|_{E \times E}$), and let $f|_E : E \to Y$ be the restriction of $f$ to $E$, thus
\[
f|_E(x) := f(x) \quad \text{when } x \in E.
\]  

If $x_0 \in E$ and $f$ is continuous at $x_0$, show that $f|_E$ is also continuous at $x_0$.  
(Is the converse of this statement true? Explain.)  

Conclude that if $f$ is continuous, then $f|_E$ is continuous.  
Thus restriction of the domain of a function does not destroy continuity.  

\emph{Hint: use part (a).}
 
 \item[(c)] 
Let $f : X \to Y$ be a function from one metric space $(X,d_X)$ to another $(Y,d_Y)$.  
Suppose that the image $f(X)$ of $X$ is contained in some subset $E \subseteq Y$ of $Y$.  
Let $g : X \to E$ be the function which is the same as $f$ but with the codomain restricted from $Y$ to $E$, thus $g(x) = f(x)$ for all $x \in X$.  

\medskip
\textbf{Note on codomain:}  
The \emph{codomain} of a function is the declared target set of the function, in contrast to the \emph{image} (or range), which is the set of values the function actually takes.  
So while $f$ is originally defined with codomain $Y$, its values all lie in the smaller set $E \subseteq Y$.  
Therefore, one can equivalently regard $f$ as a function $g : X \to E$.  
The metric on $E$ is the one \emph{induced from $Y$}, i.e.\ $d_Y|_{E \times E}$.

\medskip
Show that for any $x_0 \in X$, $f$ is continuous at $x_0$ if and only if $g$ is continuous at $x_0$.  
Conclude that $f$ is continuous if and only if $g$ is continuous.  

(Thus the notion of continuity is not affected if one restricts the codomain of the function.)

\end{enumerate}
\end{problem}
%────────────────────────────────────────────────────────────────────────────────────────────────────────────────────────────────────────────────────

\begin{problem}[20pts]
  Let $(X,d_X)$ and $(Y,d_Y)$ be metric spaces and $f:X \mapsto Y$ is a function from $X$ to $Y$.
\begin{enumerate}

  \item[(a)] 
  Prove that $f$ is continuous on $X$ if, and only if,
\[
f(\overline{A}) \subseteq \overline{f(A)}
\]
for every subset $A$ of $X$.

  \item[(b)]  Prove that $f$ is continuous on $X$ if, and only if, $f$ is continuous on every compact subset of $X$.  

\textit{Hint:} If $x_n \to p$ in $X$, the set $\{p, x_1, x_2, \dots \}$ is compact.

 \end{enumerate}
\end{problem}