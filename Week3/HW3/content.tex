\begin{problem}
    Let $(x^{(n)})_{n=m}^\infty$ be a sequence of points in a metric space $(X,d)$, and let $L\in X$. Show that if $L$ is a limit point of the sequence $(x^{(n)})_{n=m}^\infty$, then $L$ is an adherent point of the set
\[
S = \{ x^{(n)} : n\ge m \}.
\]
Is the converse true?
\end{problem}

\begin{proof}
    Suppose $L$ is a limit point of the sequence. By definition,
    \[
    \forall N \geq m, \forall \varepsilon > 0,\ \exists n \geq N \text{ such that } d(x^{(n)}, L) \leq \varepsilon.
    \]
    This implies that for every $\varepsilon > 0$, there exists $n \geq m$ with
    \[
    x^{(n)} \in B(L,\varepsilon) \cap S \neq \varnothing.
    \]
    Hence,
    \[
    \forall \varepsilon > 0,\ B(L,\varepsilon)\cap S \neq \varnothing 
    \quad \Rightarrow \quad 
    L \text{ is an adherent point of } S.
    \]

    Now, we check the converse. The converse statement is \textbf{NOT} true.  
    Consider $X = \mathbb{R}$ with the standard metric, and let $m=1$. Define
    \[
    x^{(n)} = \frac{1}{n}, \quad \text{so } S = \{1, \tfrac{1}{2}, \tfrac{1}{3}, \dots \}.
    \]
    It is clear that $1$ is an adherent point of $S$, since for every $\varepsilon > 0$,
    \[
    1 \in (1-\varepsilon, 1+\varepsilon), 
    \quad \Rightarrow \quad 
    B(1,\varepsilon)\cap S \neq \varnothing.
    \]
    However, $1$ is not a limit point of the sequence. Indeed, if $N \geq 2$, then for all $n \geq N$,
    \[
    d(x^{(n)},1) = \bigg| \frac{1}{n} - 1 \bigg| \geq \frac{1}{2}.
    \]
    So if we take $\varepsilon = 0.48763$ and $N = 2$, there is no $n \geq N$ such that $d(x^{(n)},1)\leq \varepsilon$.
    Therefore, $1$ is not a limit point of $(x^{(n)})_{n=1}^\infty$, 
    even though it is an adherent point of $S$.
\end{proof}

\begin{problem}
    The following construction generalizes the construction of the reals from the rationals in Chapter~5, allowing one to view any metric space as a subspace of a complete metric space. In what follows we let $(X,d)$ be a metric space.
\begin{enumerate}
  \item[(a)] Given any Cauchy sequence $(x_n)_{n=1}^\infty$ in $X$, we introduce the \emph{formal limit} 
  \[
  \operatorname{LIM}_{n\to\infty} x_n.
  \]
  We say that two formal limits $\operatorname{LIM}_{n\to\infty} x_n$ and $\operatorname{LIM}_{n\to\infty} y_n$ are equal if 
  \[
  \lim_{n\to\infty} d(x_n,y_n) = 0.
  \]
  Show that this equality relation obeys the reflexive, symmetry, and transitive axioms, i.e.\ that it is an equivalence relation.

  \item[(b)] Let $\overline{X}$ be the space of all formal limits of Cauchy sequences in $X$, modulo the above equivalence relation. Define a metric $d_{\overline{X}}:\overline{X}\times\overline{X}\to [0,\infty)$ by
  \[
  d_{\overline{X}}\!\left(\operatorname{LIM}_{n\to\infty}x_n, \operatorname{LIM}_{n\to\infty} y_n\right) := \lim_{n\to\infty} d(x_n,y_n).
  \]
  Show that this function is well-defined (the limit exists and does not depend on the choice of representatives) and that it satisfies the axioms of a metric. Thus $(\overline{X},d_{\overline{X}})$ is a metric space.

  \item[(c)] Show that the metric space $(\overline{X},d_{\overline{X}})$ is complete.

  \item[(d)] We identify an element $x\in X$ with the corresponding constant Cauchy sequence $(x,x,x,\dots)$, i.e.\ with the formal limit $\operatorname{LIM}_{n\to\infty} x$. Show that this is legitimate: for $x,y\in X$, 
  \[
  x=y \quad \Longleftrightarrow \quad \operatorname{LIM}_{n\to\infty} x = \operatorname{LIM}_{n\to\infty} y.
  \]
  With this identification, show that 
  \[
  d(x,y) = d_{\overline{X}}(x,y),
  \]
  and thus $(X,d)$ can be thought of as a subspace of $(\overline{X},d_{\overline{X}})$.

  \item[(e)] Show that the closure of $X$ in $\overline{X}$ is $\overline{X}$ itself. (This explains the choice of notation.)

  \item[(f)] Finally, show that the formal limit agrees with the actual limit: if $(x_n)_{n=1}^\infty$ is a Cauchy sequence in $X$ that converges in $X$, then
  \[
  \lim_{n\to\infty} x_n = \operatorname{LIM}_{n\to\infty} x_n \quad \text{in } \overline{X}.
  \]
\end{enumerate}
\end{problem}

\begin{proof}[a] 
    We verify the following properties:
    \begin{itemize}
        \item Reflexive: $\operatorname{LIM}_{n\to\infty} x_n$ and $\operatorname{LIM}_{n\to\infty} x_n$ are equal since $d$ is metric, so $\forall n, d(x_n, x_n) = 0$.
        \item Symmetry: If $\operatorname{LIM}_{n\to\infty} x_n$ and $\operatorname{LIM}_{n\to\infty} y_n$ are equal, this mean $\lim_{n\to\infty} d(x_n,y_n) = 0$. And since $d$ is metric, so $\lim_{n\to\infty} d(y_n,x_n) = 0$, hence $\operatorname{LIM}_{n\to\infty} y_n$ and $\operatorname{LIM}_{n\to\infty} x_n$ are equal.
        \item Transitive: If $\operatorname{LIM}_{n\to\infty} x_n$ and $\operatorname{LIM}_{n\to\infty} y_n$ are equal and $\operatorname{LIM}_{n\to\infty} y_n$ and $\operatorname{LIM}_{n\to\infty} z_n$ are equal, this mean $\lim_{n\to\infty} d(x_n,y_n) = 0$ and $\lim_{n\to\infty} d(y_n,z_n) = 0$, since $d$ is metric, so $0 \leq \lim_{n\to\infty} d(x_n,z_n) \leq \lim_{n\to\infty} d(x_n,y_n) + \lim_{n\to\infty} d(y_n,z_n) = 0$. By squeeze lemma, $\lim_{n\to\infty} d(x_n,z_n) = 0$, so  $\operatorname{LIM}_{n\to\infty} x_n$ and $\operatorname{LIM}_{n\to\infty} z_n$ are equal.
    \end{itemize}
\end{proof}

\begin{proof}[b]
    We verify the following properties:
    \begin{itemize}
        \item The limit existence: 
    \end{itemize}
\end{proof}

\begin{problem}
    In the following, all the sets are subsets of a metric space $(X,d)$.

 \begin{enumerate}
  \item[(a)] If $\overline{A}\cap\overline{B}=\varnothing$, then 
  \[
  \partial(A\cup B) = \partial A \cup \partial B.
  \]

  \item[(b)] For a finite family $\{A_i\}_{i=1}^n\subseteq X$, show that
  \[
  \operatorname{int}\!\Bigl(\bigcap_{i=1}^n A_i\Bigr)
  \;=\;
  \bigcap_{i=1}^n \operatorname{int}(A_i).
  \]

  \item[(c)] For an arbitrary (possibly infinite) family $\{A_\alpha\}_{\alpha\in F}\subseteq X$, prove that
  \[
  \operatorname{int}\!\Bigl(\bigcap_{\alpha\in F} A_\alpha\Bigr)
  \;\subseteq\;
  \bigcap_{\alpha\in F}\operatorname{int}(A_\alpha).
  \]

  \item[(d)] Give an example where the inclusion in part \textup{(c)} is strict (i.e., equality fails).

  \item[(e)] For any family $\{A_\alpha\}_{\alpha\in F}\subseteq M$, prove that
  \[
  \bigcup_{\alpha\in F}\operatorname{int}(A_\alpha)
  \;\subseteq\;
  \operatorname{int}\!\Bigl(\bigcup_{\alpha\in F} A_\alpha\Bigr).
  \]

  \item[(f)] Give an example of a finite collection $F$ in which equality does not hold in part \textup{(e)}.
\end{enumerate}

\end{problem}

\begin{problem}
    Let $(X, d)$ be a metric space and $Y \subset X$ be an open subset. For any subset $A \subset Y$, show
that $A$ is open in $Y$ if and only if it is open in $X$.
\end{problem}

\begin{problem}
    On the space $(0,1]$, we may consider the topology induced by the metric space $(\mathbb{R},d)$ defined by
$d(x,y)=|x-y|$ . Alternatively, we may also define a distance $d'$ on $(0,1]$, given by
\[
d'(x,y) = \left| \frac{1}{x} - \frac{1}{y} \right|, \qquad \forall x,y \in (0,1].
\]

\begin{enumerate}
 \item[(a)] Show that $d'$ is a metric on $(0,1]$
 \item[(b)] Let $x \in (0,1]$ and $\varepsilon>0$. Let $B = B_{d}(x,\varepsilon)=\{y | |y-x| < \varepsilon \} \cap (0,1]$  be the open ball centered at $x$ of radius $\varepsilon$ for the metric $d$ in $(0,1]$.  
  Show that for any $y \in B$, we may find $\varepsilon'>0$ such that
  \[
  B_{d'}(y,\varepsilon') \subseteq B = B_{d}(x,\varepsilon).
  \]

 \item[(c)]Show that an open ball in $((0,1],d')$ is also an open ball in $((0,1],d)$.

 \item[(d)] Conclude that the metric spaces $((0,1],d)$ and $((0,1],d')$ are topologically equivalent, that is, a set $A$ is open in one space if and only if it is also open in the other one.

 \item[(e)] Is $((0,1],d')$ a complete metric space? How about $((0,1],d)$?
\end{enumerate}
\end{problem}

\begin{proof}[a]
    We verify the following properties:
    \begin{itemize}
        \item $d'(x,y) \geq 0$ since $\lvert \cdot \rvert \geq 0$.
        \item $d'(x,y) = 0 \iff \lvert\frac{1}{x} - \frac{1}{y}\rvert \iff x=y$.
        \item $d'(x,y) = \lvert\frac{1}{x} - \frac{1}{y}\rvert = \lvert\frac{1}{y} - \frac{1}{x}\rvert = d'(y,x)$.
        \item We know if $a, b \in \mathbb{R}$, the triangular inequality $\lvert a \rvert + \lvert b \rvert \geq \lvert a+b \rvert$ holds.
        Then we can plug $a = \frac{1}{x} - \frac{1}{y}$ and $b = \frac{1}{y} - \frac{1}{z}$ in. Then we can get $\lvert \frac{1}{x} - \frac{1}{y} + \frac{1}{y} - \frac{1}{z} \rvert \leq \lvert \frac{1}{x} - \frac{1}{y} \rvert + \lvert \frac{1}{y} - \frac{1}{z} \rvert$. Hence $d'(x,z) \leq d'(x,y) + d'(y,z)$.
    \end{itemize}
    Hence, $d'$ is indeed a metric on $(0.1]$.
\end{proof}

\begin{proof}[b]
    We assume that $d(x,y) = \varepsilon'' < \varepsilon$.
    \begin{claim}
        We claim that if we pick $\varepsilon' = \varepsilon - \varepsilon'' > 0$, then $B_{d'}(y,\varepsilon') \subseteq B_d(x,\varepsilon)$
    \end{claim}
    \begin{explanation}
        $\forall z \in B_{d'}(y, \varepsilon')$, $\lvert \frac{1}{y} - \frac{1}{z} \rvert < \varepsilon'$. Then $\lvert \frac{z-y}{zy} \rvert < \varepsilon'$, $\lvert z-y \rvert < \varepsilon'zy$ (Since $z > 0$ and $y > 0$). \\
        Hence we know that $d(y,z) = \lvert y-z \rvert < \varepsilon'zy < \varepsilon'$. \\
        Since $d$ is metric, so $d(x,z) \leq d(x,y) + d(y,z) < \varepsilon'' + (\varepsilon - \varepsilon'') = \varepsilon$. \\
        Hence $z \in B_d(x, \varepsilon)$.
    \end{explanation}
    By claim, we done.
\end{proof}

\begin{proof}[c]
    To prove this, we need to show given $x \in (0,1], \varepsilon >0, B = B_{d'}(x, \varepsilon)$, for every $y \in B$, we can find $\varepsilon' > 0$ such that $B_d(y, \varepsilon') \subseteq B = B_{d'}(x, \varepsilon)$
    We may assume that $d'(x,y) = \varepsilon'' < \varepsilon$.
    \begin{claim}
        We claim that if we pick $\varepsilon' = ? > 0$, then then $B_{d}(y,\varepsilon') \subseteq B_{d'}(x,\varepsilon)$
    \end{claim}
    \begin{explanation}
        
    \end{explanation}
    
\end{proof}

\begin{problem}
    \begin{enumerate}

  \item[(a)] 
  We say that a family of closed balls 
\[
\bigl(\overline{B}(x_n,r_n)\bigr)_{n\ge 1}
\]
is a \emph{decreasing sequence of closed balls} if 
the nesting condition
\[
\overline{B}(x_{n+1},r_{n+1}) \;\subseteq\; \overline{B}(x_n,r_n)
\quad\text{for all } n\in\mathbb{N}
\]
is satisfied. Give an example of a decreasing sequence of closed balls in a complete metric space with empty intersection. 

  \item[(b)]  We say that a family of closed balls 
\[
\bigl(\overline{B}(x_n,r_n)\bigr)_{n\ge 1}
\]
is a \emph{decreasing sequence of closed balls with radii tending to zero} if 
\[
r_n \;\to\; 0 \quad\text{as } n\to\infty,
\]
and the nesting condition
\[
\overline{B}(x_{n+1},r_{n+1}) \;\subseteq\; \overline{B}(x_n,r_n)
\quad\text{for all } n\in\mathbb{N}
\]
is satisfied.
  Show that a metric space $(M,d)$ is complete if and only if every decreasing sequence of closed balls with radii going to zero has a nonempty intersection. \end{enumerate}
\end{problem}
