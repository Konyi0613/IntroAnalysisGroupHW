\begin{problem}[15pts \textbf{Exercise 4.7.8}]
    Let $\tan : (-\pi/2,\pi/2) \to \mathbb{R}$ be the tangent function 
$\tan(x) := \sin(x)/\cos(x)$.  
Show that $\tan$ is differentiable and monotone increasing, with  
\[
\frac{d}{dx}\tan(x) = 1 + \tan(x)^2,
\]
and that $\lim_{x\to \pi/2} \tan(x) = +\infty$ and $\lim_{x\to -\pi/2} \tan(x) = -\infty$.  
Conclude that $\tan$ is in fact a bijection from $(-\pi/2,\pi/2) \to \mathbb{R}$, and thus has an inverse function 
\[
\tan^{-1} : \mathbb{R} \to (-\pi/2,\pi/2)
\]
(this function is called the \emph{arctangent function}).  
Show that $\tan^{-1}$ is differentiable and 
\[
\frac{d}{dx} \tan^{-1}(x) = \frac{1}{1+x^2}.
\]
\end{problem}
\begin{proof}
    Since \(\sin (x)\) and \(\cos (x)\) are both differentiable on \(\left( - \frac{\pi}{2}, \frac{\pi}{2} \right) \), and \(\cos (x) \neq 0\) on \(\left( -\frac{\pi}{2}, \frac{\pi}{2} \right) \), so \(\tan (x)\) is differentiable and 
    \begin{align*}
         \frac{\mathrm{d}}{\mathrm{d}x} \tan (x) &= \frac{\mathrm{d}}{\mathrm{d}x} \frac{\sin (x)}{\cos (x)} = \frac{(\sin (x))^{\prime} \cos (x) - \sin (x) \left( \cos (x) \right)^{\prime}  }{\cos ^2(x)} 
         \\ &= \frac{\cos ^2(x) + \sin ^2(x)}{\cos ^2(x)} = 1 + \frac{\sin ^2(x)}{\cos ^2(x)} = 1 + \tan^2(x).
    \end{align*} 
    Hence, 
    \[
        \frac{\mathrm{d}}{\mathrm{d}x} \tan (x) = 1 + \tan ^2(x) \ge 1 > 0 
    \]    
    for all \(x \in \left( -\frac{\pi}{2}, \frac{\pi}{2} \right) \) and thus \(\tan (x)\) is monotone increasing. Note that 
    \[
        \lim_{x \to \frac{\pi}{2}} \tan (x) = \lim_{x \to \left( \frac{\pi}{2} \right)^- } \tan (x) = \lim_{x \to \left( \frac{\pi}{2} \right)^- } \frac{\sin (x)}{\cos (x)},
    \]  
    and \(\sin (x) \to 1\) and \(\cos (x) \to 0^+\) as \(x \to \left( \frac{\pi}{2} \right)^- \), so 
    \[
        \lim_{x \to \left( \frac{\pi}{2} \right)^- } \frac{\sin (x)}{\cos (x)} = +\infty . 
    \]  
    Similarly, since \(\sin (x) \to -1\) and \(\cos (x) \to 0^+\) as \(x \to \left( -\frac{\pi}{2} \right)^+ \), so 
    \[
        \lim_{x \to -\frac{\pi}{2}} \tan (x) = \lim_{x \to \left( -\frac{\pi}{2} \right)^+ } \tan (x) = \lim_{x \to \left( -\frac{\pi}{2} \right)^+ } \frac{\sin (x)}{\cos (x)} = -\infty.   
    \]   
    Hence, we have shown that 
    \[
        \lim_{x \to \frac{\pi}{2}} \tan (x) = +\infty , \quad \lim_{x \to \left( - \frac{\pi}{2} \right) } \tan (x) = -\infty. 
    \]
    By this, we know \(\tan (x)\) is a bijection from \(\left( -\frac{\pi}{2}, \frac{\pi}{2} \right) \to \mathbb{R}  \), and thus the inverse function \(\tan ^{-1}(x)\) is well-defined. Also, since 
    \[
        \tan \left( \tan ^{-1}(x) \right) = x, 
    \]   
    so by differentiating on both sides, we have 
    \begin{align*}
        1 &= (x)^{\prime} = \left( \tan \left( \tan ^{-1}(x) \right)  \right)^{\prime} = \tan ^{\prime} \left( \tan ^{-1}(x) \right) \cdot \left( \tan ^{-1}(x) \right)^{\prime}  \\
        &= \left( 1 + \left( \tan \left( \tan ^{-1}(x) \right)  \right)^2  \right) \cdot \left( \tan ^{-1} (x) \right)^{\prime} = (1 + x^2) \cdot \left( \frac{\mathrm{d}}{\mathrm{d}x} \tan ^{-1}(x)  \right).   
    \end{align*}
    Hence, 
    \[
        \frac{\mathrm{d}}{\mathrm{d}x} \tan ^{-1} (x) = \frac{1}{1 + x^2}. 
    \]
\end{proof}

\begin{problem}[15pts \textbf{Exercise 4.7.9}]
    Recall the arctangent function $\tan^{-1}$ from Exercise 4.7.8.  
By modifying the proof of Theorem 4.5.6(e), establish the identity
\[
\tan^{-1}(x) = \sum_{n=0}^{\infty} \frac{(-1)^n x^{2n+1}}{2n+1}
\]
for all $x \in (-1,1)$.  
Using Abel's theorem (Theorem 4.3.1) to extend this identity to the case $x=1$, conclude in particular the identity
\[
\pi = 4 - \frac{4}{3} + \frac{4}{5} - \frac{4}{7} + \cdots 
= 4 \sum_{n=0}^{\infty} \frac{(-1)^n}{2n+1}.
\]

(Note that the series converges by the alternating series test, Proposition 7.2.11.)  
Conclude in particular that $4 - \tfrac{4}{3} < \pi < 4$.  
(One can of course compute $\pi = 3.1415926\ldots$ to much higher accuracy, though if one wishes to do so it is advisable to use a different formula than the one above, which converges very slowly.)
\end{problem}

\begin{proof}
    From the previous question, we know that
    \[
    \frac{\mathrm{d}}{\mathrm{d}x} \tan ^{-1} (x) = \frac{1}{1 + x^2},
    \]
    and hence
    \[
    \frac{\mathrm{d}}{\mathrm{d}x} \tan ^{-1} (x) = \frac{1}{1 - (-x^2)} = \sum_{n=0}^{\infty} (-x^2)^n = \sum_{n=0}^{\infty} (-1)^n(x^{2n}), \quad \text{for }\lvert x\rvert < 1.
    \]
    And hence if $t \in (-1,1)$
    \[
    \int_0^t \left( \frac{\mathrm{d}}{\mathrm{d}x} \tan ^{-1} (x)\right) \mathrm{d}x = \int_0^t \left( \sum_{n=0}^{\infty} (-1)^n(x^{2n})\right) \mathrm{d}x.
    \]
    So we have
    \[
    \left.\tan ^{-1} (x) \right|_0^t = \left. \left( \sum_{n=0}^{\infty} \frac{1}{2n+1} (-1)^n(x^{2n+1}) \right)\right|_0^t
    \]
    \[
    \tan ^{-1} (t) = \sum_{n=0}^{\infty} \frac{(-1)^n(t^{2n+1})}{2n+1}
    \]
    Now replace $x = t$, then for all $x \in (-1,1)$, we can get
    \[
    \tan ^{-1} (x) = \sum_{n=0}^{\infty} \frac{(-1)^n(x^{2n+1})}{2n+1}.
    \]
    Since
    \[
    \limsup_{n \to \infty} \left \vert \frac{1}{2n+1}\right \vert^{\frac{1}{n}} = 1,
    \]
    the radius of convergence of $\sum_{n=0}^{\infty} \frac{(-1)^n(t^{2n+1})}{2n+1}$ is 1. \\
    We want to extend the identity for $x=1$, we want to show that 
    \[
    \sum_{n=0}^{\infty} \frac{(-1)^n(1^{2n+1})}{2n+1} = \sum_{n=0}^{\infty} \frac{(-1)^n}{2n+1} \text{ converge.}
    \]
    Let $a_n = \frac{1}{2n+1}$, then $a_n > 0$ for all $n$, and $\{a_n\}$ is decreasing, and $a_n \to 0$ as $n \to \infty$, So $\sum (-1)^na_n$ converges by alternating series test, and hence $\sum_{n=0}^{\infty} \frac{(-1)^n}{2n+1}$ converges. \\
    By Abel's Theorem,
    \[
    \tan ^{-1} (1) = \frac{\pi}{4} = 1 - \frac{1}{3} + \frac{1}{5} - \cdots,
    \]
    and hence
    \[
    \pi = 4 - \frac{4}{3} + \frac{4}{5} - \cdots = 4 \sum_{n=0}^{\infty} \frac{(-1)^n}{2n+1}.
    \]
    Then we show that $\frac{4}{3} < \pi < 4$. First we show the upper bound, write
    \[
    \pi = S = 4 - \Bigl(\frac{4}{3}-\frac{4}{5}+\frac{4}{7}-\frac{4}{9}+\cdots\Bigr).
    \]
    Now group the terms inside the parentheses in pairs:
    \[
    \frac{4}{3}-\frac{4}{5}+\frac{4}{7}-\frac{4}{9}+\cdots
    =
    \Bigl(\frac{4}{3}-\frac{4}{5}\Bigr)
    +\Bigl(\frac{4}{7}-\frac{4}{9}\Bigr)
    +\Bigl(\frac{4}{11}-\frac{4}{13}\Bigr)+\cdots .
    \]
    Each pair is positive because its first denominator is smaller:
    \[
    \frac{4}{3}-\frac{4}{5} > 0,\quad
    \frac{4}{7}-\frac{4}{9} > 0,\quad \ldots
    \]
    Hence the whole series in parentheses is a sum of positive terms, so it is
    \emph{positive}. Therefore
    \[
    \pi = S = 4 - (\text{positive number}) < 4.
    \]
    Then we show the lower bound, write
    \[
    \pi = S = \Bigl(4-\frac{4}{3}\Bigr)
       +\Bigl(\frac{4}{5}-\frac{4}{7}+\frac{4}{9}-\frac{4}{11}+\cdots\Bigr).
    \]
    Again, group the infinite tail in pairs:
    \[
    \frac{4}{5}-\frac{4}{7}+\frac{4}{9}-\frac{4}{11}+\cdots
    =
    \Bigl(\frac{4}{5}-\frac{4}{7}\Bigr)
    +\Bigl(\frac{4}{9}-\frac{4}{11}\Bigr)
    +\Bigl(\frac{4}{13}-\frac{4}{15}\Bigr)+\cdots .
    \]
    Each of these pairs is also positive, e.g.
    \[
    \frac{4}{5}-\frac{4}{7} > 0,\quad
    \frac{4}{9}-\frac{4}{11} > 0,\quad \ldots
    \]
    so every partial sum of this tail is positive, and hence its limit (the value
    of the tail) is a positive number. Thus
    \[
    \pi = S = \Bigl(4-\frac{4}{3}\Bigr) + (\text{positive number})
        > 4-\frac{4}{3}.
    \]
    Combining the two inequalities, we can prove that
    \[
    4 - \frac{4}{3} < \pi < 4.
    \]
\end{proof}

\begin{problem}[30pts \textbf{Exercise 4.7.10}]
    Let $f : \mathbb{R} \to \mathbb{R}$ be the function
\[
f(x) := \sum_{n=1}^{\infty} 4^{-n} \cos(32^n \pi x).
\]

\begin{enumerate}
\item[(a)] Show that this series is uniformly convergent, and that $f$ is continuous.

\item[(b)] Show that for every integer $j$ and every integer $m \ge 1$, we have
\[
\left| f\!\left( \frac{j+1}{32^m} \right) - 
       f\!\left( \frac{j}{32^m} \right) \right| 
\ge 4^{-m}.
\]


\textit{Hint: use the identity}
\[
\sum_{n=1}^{\infty} a_n 
= \left( \sum_{n=1}^{m-1} a_n \right)
+ a_m 
+ \sum_{n=m+1}^{\infty} a_n
\]
\textit{for certain sequences } $a_n$.  
Also, use the fact that the cosine function is periodic with period $2\pi$, as well as the geometric series formula  
$\sum_{n=0}^{\infty} r^n = \frac{1}{1-r}$ for any $|r|<1$.  
Finally, you will need the inequality $|\cos(x)-\cos(y)| \le |x-y|$ for any real numbers $x$ and $y$; this can be proven by using the mean value theorem.

\item[(c)] Using (b), show that for every real number $x_0$, the function $f$ is not differentiable at $x_0$.  
(Hint: for every $x_0$ and every $m \ge 1$, there exists an integer $j$ such that 
$j \le 32^m x_0 \le j+1$, thanks to Exercise 5.4.3.)

\item[(d)] Explain briefly why the result in (c) does not contradict Corollary 3.7.3.
\end{enumerate}
\end{problem}

\begin{proof}[(a)]

Let \( M_n = 4^{-n} \). Then
\[
\sum_{n=1}^\infty M_n = \sum_{n=1}^\infty 4^{-n} < \infty.
\]
Since
\[
|4^{-n}\cos(32^n \pi x)| \le 4^{-n} = M_n,
\]
the Weierstrass \(M\)-test implies that
\[
\sum_{n=1}^\infty 4^{-n}\cos(32^n \pi x)
\]
converges uniformly on \(\mathbb{R}\). 
\\

Since each term \(4^{-n}\cos(32^n\pi x)\) is continuous, hence every partial sum
\[
f_N(x) = \sum_{n=1}^N 4^{-n}\cos(32^n\pi x)
\]
is continuous.  
Since \(f_N \to f\) uniformly and each \(f_N\) is continuous, the limit \(f\) is continuous.

\end{proof}
\begin{proof}[(b)]
Let
\[
x=\frac{j}{32^m}, \qquad y=\frac{j+1}{32^m}.
\]
Then
\[
f(y)-f(x)
= \sum_{n=1}^\infty 4^{-n} 
\left[ \cos(32^n\pi y) - \cos(32^n\pi x) \right].
\]

Split the sum at \(n=m\):
\begin{align*}
f(y)-f(x)
=&
\underbrace{
\sum_{n=1}^{m-1} 4^{-n} 
\left[ \cos(32^n\pi y) - \cos(32^n\pi x) \right]
}_{A}\\
+&
\underbrace{
\sum_{n=m+1}^{\infty} 4^{-n} 
\left[ \cos(32^n\pi y) - \cos(32^n\pi x) \right]
}_{B}\\
+&
\underbrace{
4^{-m} \left[ \cos(\pi(j+1)) - \cos(\pi j) \right]
}_{C}.
\end{align*}

Since \(\cos\) is \(2\pi\)-periodic,
\[
B = \sum_{n=m+1}^{\infty}
4^{-n}\left[\cos(32^n\pi y) - \cos(32^n\pi x)\right]
= 0.
\]

Next compute
\[
C = 4^{-m}\left[(-1)^{j+1} - (-1)^j\right]
    = 4^{-m}\bigl[-2(-1)^j\bigr],
\]
so
\[
|C| = 2 \cdot 4^{-m}.
\]

\begin{align*}
|A|
&= \left|
\sum_{n=1}^{m-1} 4^{-n}
\bigl(
\cos(32^n\pi y)-\cos(32^n\pi x)
\bigr)
\right| \\[4pt]
&\le \sum_{n=1}^{m-1} 4^{-n} 
\left|\cos(32^n\pi y)-\cos(32^n\pi x)\right| \\[4pt]
&\le \sum_{n=1}^{m-1} 4^{-n} 
\left| 32^n\pi y - 32^n\pi x \right| \\[4pt]
&= \sum_{n=1}^{m-1} 4^{-n} \pi 32^n |y-x| \\[4pt]
&= \pi |y-x| \sum_{n=1}^{m-1} 8^n \\[4pt]
&= \pi 32^{-m} \sum_{n=1}^{m-1} 8^n \\[4pt]
&< \pi 32^{-m} \cdot \frac{8^m}{7} \\[4pt]
&= \frac{\pi}{7} \cdot \frac{8^m}{32^m}
 = \frac{\pi}{7}\left(\frac{1}{4}\right)^m
 = \frac{\pi}{7}\,4^{-m}.
\end{align*}

Hence

\begin{align*}
|f(y)-f(x)| 
&= |A + C|\\
&\ge |C| - |A|\\
&\ge 2 \cdot 4^{-m} - \frac{\pi}{7}\,4^{-m}\\
&\ge 4^{-m}
\end{align*}

\paragraph{Note.}
Prove
\[
|\cos(x)-\cos(y)| \le |x-y| \qquad \forall x,y\in\mathbb{R}.
\]

By mean value theorem, since $\cos$ is continuous and differentiable on $\mathbb{R}$,
\[
\exists\, c \text{ such that } 
\cos'(c) = -\sin(c)
        = \frac{\cos(x)-\cos(y)}{x-y}.
\]

Therefore,
\[
|\sin(c)|\,|x-y|
= |\cos(x)-\cos(y)|.
\]

Since $|\sin(c)|\le 1$, we get
\[
|x-y| \ge |\cos(x)-\cos(y)|.
\]

\end{proof}
\begin{proof}[(c)]

Let 
\[
a_m = \frac{j_m}{32^m}, \qquad 
b_m = \frac{j_m+1}{32^m}
\qquad (j_m \in \mathbb{N}).
\]

Suppose $\exists\, j_m$ such that $f'\!\left(\frac{j_m}{32^m}\right)$ exists.  
Then
\[
f'\!\left(\frac{j_m}{32^m}\right)
= \lim_{h\to 0} 
\frac{f\!\left(\frac{j_m}{32^m}+h\right) - 
      f\!\left(\frac{j_m}{32^m}\right)}{h}
\quad \text{exists}.
\]

Take $h = 32^{-m}$.  
Then
\[
\lim_{m\to\infty}
\frac{f(b_m) - f(a_m)}{b_m - a_m}
\quad \text{exists}.
\]

\medskip

For all $m\ge 1$ and for all $x_0\in\mathbb{R}$,  
by Ex.\ 5.4.3, we have  
\[
a_m \le x_0 \le b_m.
\]

By part (b),
\[
\left|\frac{f(b_m) - f(a_m)}{b_m - a_m}\right|
\ge
\frac{4^{-m}}{32^{-m}}
= 8^m.
\]

Thus
\[
\lim_{m\to\infty}
\frac{f(b_m) - f(a_m)}{b_m - a_m}
\quad \text{does not exist}.
\]

Hence we obtain a contradiction.  
Therefore, for all $x_0$,  
\[
f'(x_0) \text{ does not exist}.
\]



\end{proof}
\begin{proof}[(d)]

Corollary 3.7.3 requires that
\[
\sum \|f'_n\|_\infty < \infty.
\]
But here, the condition of the corollary does not apply.
\end{proof}

\begin{problem}[20pts]
\vphantom{text}
    \begin{enumerate}

\item[(a)]  Prove that \[
(\cos\theta + i\sin\theta)^{n}
= \cos(n\theta) + i\sin(n\theta)
\] or all integers $n$ and all real $\theta$.
This is the classical \emph{DeMoivre’s theorem}.

\item[(b)]
By equating imaginary parts in DeMoivre's formula, prove that
\[
\sin n\theta
= \sin^n \theta \left\{
\binom{n}{1} \cot^{\,n-1}\theta
- \binom{n}{3} \cot^{\,n-3}\theta
+ \binom{n}{5} \cot^{\,n-5}\theta
-\ \cdots
\right\}.
\]

\item[(c)]
If $0<\theta<\pi/2$, prove that
\[
\sin(2m+1)\theta
= \sin^{\,2m+1}\!\theta \; P_m(\cot^2\theta)
\]
where $P_m$ is the polynomial of degree $m$ given by
\[
P_m(x)
= \binom{2m+1}{1} x^m
- \binom{2m+1}{3} x^{m-1}
+ \binom{2m+1}{5} x^{m-2}
- \cdots .
\]

Use this to show that $P_m$ has zeros at the $m$ distinct points
\[
x_k = \cot^2\!\left(\frac{\pi k}{2m+1}\right),
\qquad k = 1,2,\dots,m.
\]

\item[(d)]
Show that the sum of the zeros of $P_m$ is given by
\[
\sum_{k=1}^m \cot^2\!\left(\frac{\pi k}{2m+1}\right)
= \frac{m(2m-1)}{3}.
\]
\end{enumerate}
\end{problem}

\begin{proof}[(a)]
    For fixed $\theta$, we can do induction on $n$.
    \begin{itemize}
        \item Base case $(n = 1)$: Trivial.
        \item Induction step $(n = k)$: Suppose it is true.
        \item Induction hypothesis $(n = k+1)$: 
        \[
        \begin{aligned}
        (\cos\theta + i\sin\theta)^{n+1} &= (\cos\theta + i\sin\theta)^{n}(\cos\theta + i\sin\theta) \\
        &= (\cos (n\theta) + i\sin(n\theta))(\cos\theta + i\sin\theta) \\
        &= \cos (n\theta)\cos\theta + i\sin(n\theta)\cos\theta + i\cos (n\theta)\sin\theta - \sin(n\theta)\sin\theta\\
        &= (\cos (n\theta)\cos\theta- \sin(n\theta)\sin\theta) +i(\sin(n\theta)\cos\theta + \cos (n\theta)\sin\theta) \\
        &= \cos((n+1)\theta) + i\sin((n+1)\theta)
        \end{aligned}
        \]
    \end{itemize}
    And we can do same procedure for all $\theta \in \mathbb{R}$, and hence we proved.
\end{proof}

\begin{proof}[(b)]
    By binomial theorem, we know that
    \[
    (\cos\theta + i\sin\theta)^n = \sum_{k=0}^{n} \binom{n}{k} (\cos\theta)^{n-k} (i^k)(\sin\theta)^k 
    \]
    And by \emph{DeMoivre’s theorem}, we know that
    \[
    (\cos\theta + i\sin\theta)^n = \cos(n\theta) + i\sin(n\theta)
    \]
    So we have
    \[
    \cos(n\theta) + i\sin(n\theta) = \sum_{k=0}^{n} \binom{n}{k} (\cos\theta)^{n-k} (i^k)(\sin\theta)^k 
    \]
    By equating imaginary parts, we have
    \[
    \begin{aligned}
    \sin(n\theta) &= \binom{n}{1}(\cos\theta)^{n-1}\sin\theta - \binom{n}{3} (\cos\theta)^{n-3}(\sin\theta)^3 + \binom{n}{5} (\cos\theta)^{n-5}(\sin\theta)^5 - \cdots \\
    &= (\sin\theta)^n \left( \binom{n}{1}(\cos\theta)^{n-1}(\sin\theta)^{-(n-1)} - \binom{n}{3} (\cos\theta)^{n-3}(\sin\theta)^{-(n-3)} + \cdots \right) \\
    &= (\sin\theta)^n \left( \binom{n}{1}(\cot\theta)^{n-1} - \binom{n}{3} (\cot\theta)^{n-3} + \binom{n}{5} (\cot\theta)^{n-5} - \cdots \right)
    \end{aligned}
    \]
    And we done.
\end{proof}

\begin{proof}[(c)]
    From (b), we know that
    \[
    \sin((2m+1)\theta) = \sin^{2m+1} \theta \left( \binom{2m+1}{1} (\cot^{2}\theta)^m
    - \binom{2m+1}{3} (\cot^{2}\theta)^{m-1}
    + \cdots\right)
    \]
    And we let $x = \cot^{2}\theta$, we can get 
    \[
    \sin((2m+1)\theta) = \sin^{2m+1} \theta \left( \binom{2m+1}{1} x^m
    - \binom{2m+1}{3} x^{m-1}
    + \cdots\right)
    \]
    So it is obvious that
    \[
     \sin((2m+1)\theta) = \sin^{2m+1} \theta P_m(\cot^{2} \theta) = \sin^{2m+1}\theta P_m(x)
    \]
    Since $\sin \phi = 0$ if $\phi = k \pi, k \in \mathbb{Z}$, so if $\theta = \frac{k \pi}{2m+1}, k \in \mathbb{Z}$, $\sin (2m+1) \theta = 0$, and cotangent is strictly decreasing function in $(0, \pi)$ and it is periodic function with period $\pi$, so we can conclude that
    \[
    x_k = \cot^2\!\left(\frac{\pi k}{2m+1}\right),
    \qquad k = 1,2,\dots,m.
    \]
    are $m$ distinct root of $P_m(x)$ since all of such $x_k$, $\sin^{2m+1} \theta \ne 0$ and force $P_m(x_k) = 0$.
\end{proof}

\begin{proof}[(d)]
    By Vieta's formulas,
    \[
    x_1 + x_2 + \cdots + x_m = - \frac{a_{m-1}}{a_m},
    \]
    where $\{x_k\}$ are $m$ distinct roots of $P_m(x)$, $a_{m-1}$ is the coefficient of $x^{m-1}$ of $P_m(x)$, and $a_{m}$ is the coefficient of $x^{m}$ of $P_m(x)$. \\
    And hence
    \[
    \sum_{k=1}^{m} \cot^2(\frac{\pi k}{2m+1}) = \sum_{k=1}^{m} x_k = -\frac{-\binom{2m+1}{3}}{\binom{2m+1}{1}} = \frac{(2m)(2m-1)}{3 \cdot 2} = \frac{m(2m-1)}{3}
    \]
    And we done.
\end{proof}

\begin{problem}[20pts]
    This exercise outlines a simple proof of the formula $\zeta(2)=\sum_{n=1}^{\infty}\frac{1}{n^2}=\pi^2/6$.
Start with the inequality
\[
\sin x < x < \tan x, \qquad 0<x<\frac{\pi}{2},
\]
take reciprocals, and square each member to obtain
\[
\cot^2 x < \frac{1}{x^2} < 1 + \cot^2 x.
\]
Now put $x = \dfrac{k\pi}{2m+1}$, where $k$ and $m$ are integers with $1 \le k \le m$,
and sum on $k$ to obtain
\[
\sum_{k=1}^m \cot^2\!\left( \frac{k\pi}{2m+1} \right)
< \frac{(2m+1)^2}{\pi^2} \sum_{k=1}^m \frac{1}{k^2}
< m + \sum_{k=1}^m \cot^2\!\left( \frac{k\pi}{2m+1} \right).
\]
\medskip
Use the formula in problem 4(d) to deduce the inequality
\[
\frac{m(2m-1)\pi^2}{3(2m+1)^2}
< \sum_{k=1}^m \frac{1}{k^2}
< \frac{2m(m+1)\pi^2}{3(2m+1)^2}.
\]
Now let $m\to\infty$ to obtain $\zeta(2) = \pi^2/6$.
\end{problem}
\begin{proof}
\begin{claim}
    \(\sin x < x < \tan x\) for \(0 < x < \frac{\pi}{2}\). 
\end{claim}
\begin{explanation}
    If \(0 < x < \frac{\pi}{2}\), then 
    \[
        \sin x = \int _0^x \cos t \, \mathrm{d} t < \int _0^x 1 \, \mathrm{d} t = x  
    \]
    and 
    \[
        \tan (x) = \int _0^x \sec ^2(t) \, \mathrm{d} t = \int _0^x \frac{1}{\cos ^2(t)} \, \mathrm{d} t > \int_0^x 1 \, \mathrm{d} t = x,   
    \]
    so we have 
    \[
        \sin x < x < \tan x.
    \]
\end{explanation}
Take reciprocals, and square each member, we have 
\[
    \cot ^2 x < \frac{1}{x^2} < 1 + \cot ^2x \text{ for all } 0 < x < \frac{\pi}{2}.
\]
Now put \(x = \frac{k \pi }{2m + 1}\), where \(k\) and \(m\) are integers with \(1 \le k \le m\), then we know for all such \(x\) we have \(0 < x < \frac{\pi}{2}\), so we can sum up all \(k\) to obtain 
\[
    \sum_{k=1}^m \cot ^2 \left( \frac{k \pi }{2m + 1} \right) < \frac{(2m + 1)^2}{\pi ^2} \sum_{k=1}^m \frac{1}{k^2} < m + \sum_{k=1}^m \cot ^2 \left( \frac{k \pi }{2m + 1} \right),     
\]      
and by 4(d) we know 
\[
    \sum_{k=1}^m \cot ^2 \left( \frac{\pi k}{2m + 1} \right) = \frac{m(2m - 1)}{3},  
\]
so apply this formula we have 
\[
    \frac{m(2m - 1)\pi ^2}{3(2m + 1)^2} < \sum_{k=1}^m \frac{1}{k^2} < \frac{2m(m+1) \pi ^2}{3(2m+1)^2}, 
\]
and since 
\[
    \lim_{m \to \infty} \frac{m(2m - 1)\pi ^2}{3(2m + 1)^2} = \lim_{m \to \infty} \frac{(4m - 1)\pi ^2}{12(2m + 1)} = \lim_{m \to \infty} \frac{4 \pi ^2}{24} = \frac{\pi ^2}{6}  
\]
and 
\[
    \lim_{m \to \infty} \frac{2m(m+1)\pi ^2}{3(2m + 1)^2} = \lim_{m \to \infty} \frac{(4m + 2)\pi ^2}{12(2m + 1)} = \lim_{m \to \infty} \frac{4 \pi ^2}{24} = \frac{\pi ^2}{6}   
\]
by L'Hôpital's rule, so by Squeeze theorem we know 
\[
    \zeta (2) = \lim_{m \to \infty} \sum_{k=1}^m \frac{1}{k^2} = \frac{\pi ^2}{6}. 
\]
\end{proof}